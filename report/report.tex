% Created 2023-04-29 Sat 10:50
% Intended LaTeX compiler: pdflatex
\documentclass[10pt]{article}
\usepackage[utf8]{inputenc}
\usepackage[T1]{fontenc}
\usepackage{graphicx}
\usepackage{longtable}
\usepackage{wrapfig}
\usepackage{rotating}
\usepackage[normalem]{ulem}
\usepackage{amsmath}
\usepackage{amssymb}
\usepackage{capt-of}
\usepackage{hyperref}
\usepackage{stmaryrd}
\usepackage{minted}
\usepackage{simpleConference}
\usepackage[style=alphabetic]{biblatex}
\newcommand{\llb}{\ensuremath{\llbracket}}
\newcommand{\rrb}{\ensuremath{\rrbracket}}
\author{Luca Zaninotto -- 2057074}
\date{\today}
\title{GR(1) Verification}
\hypersetup{
 pdfauthor={Luca Zaninotto -- 2057074},
 pdftitle={GR(1) Verification},
 pdfkeywords={},
 pdfsubject={},
 pdfcreator={Emacs 28.2 (Org mode 9.5.5)}, 
 pdflang={English}}
\begin{document}

\maketitle
\section*{General reactivity of rank 1}
\label{sec:org42701e0}
In temporal logic, general reactivity formulas of rank 1 are
formulas in the shape

\[(\square\diamondsuit f_1 \wedge \square\diamondsuit f_2 \wedge \dots \wedge
  \square\diamondsuit f_n) \rightarrow (\square\diamondsuit g_1 \wedge
  \square\diamondsuit g_2 \wedge \dots \wedge \square\diamondsuit g_n)\]

In other words

\[\left(\bigwedge_{i=1}^n\square\diamondsuit f_i\right) \rightarrow
  \left(\bigwedge_{i=1}^n \square \diamondsuit g_i\right)\]

To prove that a given model respects such specifications we can see
it in another way, and verify weather the model does \emph{not} satisfy

\[\left(\bigwedge_{i=1}^n\square\diamondsuit f_i\right) \wedge
  \left(\bigvee_{i=1}^n\diamondsuit\square (\neg g_i)\right)\]

By seeing the problem this way we have a clearer way to implement an
algorithm to find weather a given specification is satisfied or not,
following the general scheme:
\begin{itemize}
\item find the loops respecting \(\square\diamondsuit f_1\), from that, find
the loops respecting \(\square\diamondsuit f_2\), and so on, until you
either found a loop satisfying \(\bigwedge_{i=1}^n\square\diamondsuit
    f_i\) or an empty set.
\begin{itemize}
\item If an empty set was found, the property is respected, the
hypothesis is false.
\end{itemize}
\item if not, for each \(i \in \{1,\dots,n\}\), starting from the loops
respecting \(\bigwedge_{i=1}^n\square\diamondsuit f_i\) search for
a loop respecting \(\diamondsuit\square \neg g_i\)
\begin{itemize}
\item if any of such loop is found, return a counterexample
\end{itemize}
\end{itemize}

Let's look at each individual step and its implementation to see how
it works.

\section*{Loops respecting \(\square\diamondsuit f_n\)}
\label{sec:org6f82694}
To find each the loop respecting \(\square\diamondsuit f_n\) we use
the algorithm for repeatability check we saw in class:

\begin{minted}[]{python}
def GF(spec, reach):
    """check weather the model, in the `reach` subset verifies
    G ( F (spec)).
    If it does, it returns the set of frontiers calculated
    """

    fsm = pynusmv.glob.prop_database().master.bddFsm
    if not reach:
	return None
    recur = reach * spec
    while (fsm.count_states(recur) != 0):
	reach = pynusmv.dd.BDD.false()
	new = fsm.pre(recur)
	news = [new, recur]
	while (fsm.count_states(new) != 0):
	    reach = reach + new 
	    if recur.entailed(reach):
		return news
	    new = (fsm.pre(new)) - reach
	    news = [new] + news
	recur = recur * reach
    return None
\end{minted}

The implementation works by constructing subsequent frontiers for
which the pre-image eventually returns to some subset of the initial
\texttt{recur} set. If at some point no such subset is found (the subset is
the empty set) then there is no loop that satisfies
\(\square\diamondsuit f_n\) starting from the \texttt{reach} subset,
otherwise the explored frontiers are returns (from which we can
build the set of loops).  To find the intersection of the two we
have to search for loops that both respect \(\square\diamondsuit
  f_n\) and \(\square\diamondsuit f_{n+1}\). In order to achieve this
the simple intersection of the sets is not enough, we have to re-run
the algorithm that checks \(\square\diamondsuit f_n\) on the loop
set built starting from the frontiers of the last run, this way we
find a new set of frontiers, contained in the first one that also
respects the second one:

\begin{minted}[]{python}
fsm = pynusmv.glob.prop_database().master.bddFsm

# find fs and gs from formula
fs, gs = parse_gr1(spec)

loop_set_f = reachability(fsm)
for f in fs:
    bdd = spec_to_bdd(fsm, f)
    fronts_f = GF(bdd, loop_set_f)
    loop_set_f = loops(fronts_f)
\end{minted}
The first formula is checked in the reachable set (as usual), the
second starting from the loop set of the first one, and so on, until
all the formulas \(f\) are checked. Once this step is done either
\texttt{loop\_set\_f} contains a valid loop set, which means that the
hypothesis is \texttt{True}, therefore we have to check for the conclusion;
or the set is empty, therefore the hypothesis is false and the
formula is respected: we can simply return \texttt{(True, None)}.

In order to execute this step we have to define the function
\texttt{loops}, building the set of all loops starting from the explored
frontiers (removing all the unwanted paths in the computation of
\(\square\diamondsuit f_n\)):

\begin{minted}[]{python}
def loops(frontiers):
    """Returns the /reach/ set build from the frontiers given as
    input, this is useful in the forward phase to restrict more and
    more the set /reach/ in the FG and GF algotihms

    """
    if not frontiers:
	return None
    fsm = pynusmv.glob.prop_database().master.bddFsm
    frontiers = list(reversed(frontiers))
    reach = frontiers[0]
    new = frontiers[0]
    for el in frontiers[1:]:
	new = fsm.post(new) * el
	new = new - reach
	reach = reach + new
    return reach
\end{minted}
Essentially since the frontiers are computed using the pre-image of
sets, the loop set is built ``going forward'', using the post-image.

\section*{Loops respecting \(\diamondsuit\square g_n\)}
\label{sec:org46b46ad}
If we arrived in this step, it means there are loops respecting
\(\bigwedge_{i=1}^n\square\diamondsuit f_i\). Starting from these
loops, we have to find some loop for which one of the \(g_i\)
holds. Note that the starting point is \emph{always} the loops respecting
\(\bigwedge_{i=1}^n\square\diamondsuit f_i\), if some loop
respecting \(\diamond\square g_i\) for some \(i\in \{0,\dots,n\}\),
this means that the formula is not respected: returning the
intermediate exploration frontiers helps us build the
counterexample.

\begin{minted}[]{python}
def FG(spec, reach, recur):
    """Check weather the model, in the `reach` subset verifies
    F ( G(spec))
    If it does, it returns the set of frontiers calculated
    """

    fsm = pynusmv.glob.prop_database().master.bddFsm
    # reach = reachability(fsm)
    if not reach:
	return None
    recur = recur * spec
    while (fsm.count_states(recur) != 0):
	reach = pynusmv.dd.BDD.false()
	new = fsm.pre(recur) * spec
	news = [new, recur]
	while (fsm.count_states(new) != 0):
	    reach = reach + new 
	    if recur.entailed(reach): # recur == reach
		return news
	    new = (fsm.pre(new) * spec) - reach
	    news = [new] + news
	recur = recur * reach
    return None
\end{minted}

We can notice it is essentially the same algorithm to check
\(\square\diamondsuit \varphi\), but in this case each frontier is
calculated also intersecting with the property we're checking.

\section*{Building the counterexample}
\label{sec:org92e669f}
If the latter step results in something different from an empty set,
it means that exists some loop that satisfies \[\exists j \mid
  \bigwedge_{i=1}^n f_i \wedge \neg g_j\] This means we can build a
witness for the falsehood of the initial implication
\[\bigwedge_{i=1}^n f_i \to \bigwedge_{i=1}^n g_i\] Starting from
the frontiers returned by the \texttt{FG} algorithm we can output a loop
with a variation of the algorithm seen in class:

\begin{enumerate}
\item compute \texttt{recur} and \texttt{reach} from the frontiers set, pick one state
in \texttt{recur}, will be our first guess for the initial loop state
\texttt{s}.
\item compute subsequent frontiers of new states based on the
post-image of the latter one (starting from \texttt{s}), until there are
no states in the last frontier, keeping track of all the union of
all new frontiers \texttt{r}.
\begin{itemize}
\item If \texttt{s} is not inside \texttt{r} pick another state in the intersection
between \texttt{r} and \texttt{reach}, repeat step 2.
\end{itemize}
\item build the loop based on the frontiers, starting from \texttt{s}:
\begin{itemize}
\item compute the pre-image of the currently considered node,
intersect it with the frontier built on the post-image (call
\texttt{pred} the intersection)
\item select one node in the \texttt{pred} set
\item expand the loop with the new node
\item repeat for all frontiers
\end{itemize}
\end{enumerate}


\begin{minted}[]{python}
def counterexample(frontiers):
    """Given a list of frontiers of the FG algorithm, build
    a lasso-shaped execution that starts from the model initial states
    and loops over the states given by the frontiers

    """
    fsm = pynusmv.glob.prop_database().master.bddFsm
    recur = frontiers[-1]
    frontiers = frontiers[:-1]
    reach = reduce(lambda x,acc: x+acc, frontiers)
    s = fsm.pick_one_state(recur)
    while True:
	r = pynusmv.dd.BDD.false()
	new = fsm.post(s) * reach
	r_front = [new]
	while not fsm.count_states(new) == 0:
	    r = r + new
	    new = (fsm.post(new) * reach) - r
	    r_front = r_front + [new]
	r = r * reach
	if s.entailed(r):
	    i = 0
	    for front in r_front:
		if s.entailed(front):
		    break
		i += 1
	    head = head_to(s)[:-1] # forward [init ... s]
	    # build the loop, in a /reverse/ order (with the pre-image)
	    loop = [s]
	    curr = s
	    for new in reversed(range(i)):
		pred = fsm.pre(curr) * r_front[new]
		curr = fsm.pick_one_state(pred)
		loop = loop + [curr]
	    loop = loop + [s]
	    # head = [init ... ], loop = [s ... s] // reversed
	    path = head + list(reversed(loop))
	    return path
	else:
	    s = fsm.pick_one_state(r)
\end{minted}

Essentially working like the algorithm shown in class, with the only
exception that it is initially reconstructing the recur and reach
sets from the frontiers. It also returns the whole lasso-shaped
execution, starting from an initial node and leading to the loop. We
can do this because the frontiers built by the \texttt{FG} algorithm ensure
that all the loops inside the \texttt{reach} set respect \(\square \neg
  g_i\) for some \(i \in \{i,\dots,n\}\).

\section*{On the implementation}
\label{sec:orgc81274d}
The implementation uses in some parts of the code the \texttt{functools}
package. It should be included with the latest releases of python,
but in case it is not, the user can install it with:
\begin{minted}[]{sh}
pip install functools
\end{minted}
\end{document}